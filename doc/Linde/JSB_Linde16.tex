% http://korpusy.klf.uw.edu.pl/doc/index.html
%%%%%%%%%%%%%%%%%%%%%%% file typeinst.tex %%%%%%%%%%%%%%%%%%%%%%%%%
%
% This is the LaTeX source for the instructions to authors using
% the LaTeX document class 'llncs.cls' for contributions to
% the Lecture Notes in Computer Sciences series.
% http://www.springer.com/lncs       Springer Heidelberg 2006/05/04
%
% It may be used as a template for your own input - copy it
% to a new file with a new name and use it as the basis
% for your article.
%
% NB: the document class 'llncs' has its own and detailed documentation, see
% ftp://ftp.springer.de/data/pubftp/pub/tex/latex/llncs/latex2e/llncsdoc.pdf
%
%%%%%%%%%%%%%%%%%%%%%%%%%%%%%%%%%%%%%%%%%%%%%%%%%%%%%%%%%%%%%%%%%%%


\documentclass[runningheads,a4paper]{llncs}

\usepackage{amssymb}
\setcounter{tocdepth}{3}
\usepackage{graphicx}

\usepackage[utf8]{inputenx}
\usepackage[T1]{fontenc}
\usepackage{lmodern}

\usepackage{url}
\urldef{\mailjsb}\path|jsbien@uw.edu.pl|
\urldef{\mailsa}\path|{jsbien,|
\urldef{\mailsb}\path|r.moszczynski}@uw.edu.pl|

\newcommand{\keywords}[1]{\par\addvspace\baselineskip
\noindent\keywordname\enspace\ignorespaces#1}

\begin{document}

\mainmatter  % start of an individual contribution

% first the title is needed
\title{Setting up a DjVu search engine\\ for scanned documents with OCR\\A case study of Linde's dictionary}

% a short form should be given in case it is too long for the running head
\titlerunning{Setting up a DjVu search engine for scanned documents with OCR}

% the name(s) of the author(s) follow(s) next
%
% NB: Chinese authors should write their first names(s) in front of
% their surnames. This ensures that the names appear correctly in
% the running heads and the author index.
%
%\author{Janusz S. Bień \and Radosław Moszczyński}
\author{Janusz S. Bień}
%
\authorrunning{Setting up a DjVu search engine for scanned documents with OCR}
% (feature abused for this document to repeat the title also on left hand pages)

% the affiliations are given next; don't give your e-mail address
% unless you accept that it will be published
\institute{Formal Linguistics Department, University of Warsaw,\\ Dobra 55, 00-312 Warszawa, Poland\\
\mailjsb\\
% \mailsa
% \mailsb\\
\url{http://www.klf.uw.edu.pl}}

%
% NB: a more complex sample for affiliations and the mapping to the
% corresponding authors can be found in the file "llncs.dem"
% (search for the string "\mainmatter" where a contribution starts).
% "llncs.dem" accompanies the document class "llncs.cls".
%

% \toctitle{Lecture Notes in Computer Science}
% \tocauthor{Authors' Instructions}
\maketitle


\begin{abstract}
  It is now standard that scans published in digital libraries are
  accompanied by dirty OCR.  It is also more and more common to
  provide a search engine for the OCRed texts. Some search engines
  highlight the hits on the scans, which is very useful. We intend to
  demonstrate that owing to the DjVu format and the Open Source
  search engine presented such functionality can be provided very
  easily and without additional costs.  \keywords{DjVu, OCR, search
    engine, regular expressions, Poliqarp}
\end{abstract}


\section{Introduction}

Although the DjVu format is renowned primarily for its high
compression ratio, this is not its most important advantage in the
present day, as this factor is no longer of crucial importance for
desktop users, and even smartphones can handle large files without
technical problems.

The unique feature of the DjVu format is the ability to address with
a URL not only a document and not only a page (the latter is at least
theoretically possible also with the PDF format), but also a page
fragment in a specific view. This allows the users, especially the
researches, to easily “quote” the relevant fragments of scanned
words. 
% An example of an extensive use of this technique is the
% Ph.D. thesis \textit{The analysis and lexicographical description of
%   Linde's dictionary for digitization purposes} \cite{JB13dr}.

A search engine for DjVu documents has been developed under the
supervision of the present author and has been in use since 2009 at
\url{http://corpora.klf.uw.edu.pl/en/}; the usage statistics can be
found at \url{http://www.klf.uw.edu.pl/} (the \textit{Słowniki}
section).

The system in question is based on the linguistically oriented
Poliqarp system (Polyinterpretation Indexing Query and Retrieval
Processor, \url{http://poliqarp.sourceforge.net/}). It has a
client-server architecture, and its Web client is available in two
variants: a text only one used for the Polish National Corpus
(\url{https://bitbucket.org/jwilk/marasca}), and a graphical one,
which uses the aforementioned feature of DjVu to highlight hits on
the scans (\url{https://bitbucket.org/jwilk/marasca-wbl}); we call it
simply Poliqarp for DjVu.

The Poliqarp query language has been inspired by the Corpus Query
Processor, a component of the Corpus Workbench developed at the
University of Stuttgart and allows for sophisticated use of regular
expressions, cf. e.g. \cite{etal04}. A power user can use them to
circumvent the OCR errors or historical spelling differences, as
illustrated e.g. in \cite{JSB2014CS}.

More background information can be found in
\cite{JSB2011LNCS}. Although the primary purpose of the system is to
be run on a server, it can be also installed on a typical desktop
using a GNU/Linux operating system. The technical documentation is
available at \url{http://korpusy.klf.uw.edu.pl/doc/index.html}.

All the required software is provided on the basis of the GNU General
Public License or another open source license.

\section{Creating graphic page images}
\label{sec:creat-page-graph}



\subsection{Scanning}
\label{sec:scanning}

scanhelper

\url{http://jwilk.net/software/scanhelper}

scan tailor

\url{http://scantailor.org/}

niektóre wakaty nadal zwichrowane!

OCR not yet

\subsection{Scan adjustments}
\label{sec:scan-adjustments}

The frontmatter added by the reprint editor distorted the page numbers
in the volumes with single pagination. It appeared also that some
original fontmatter (bastard title\footnote{the bastard title, usually
  a single line in capital letters, precedes the full title, and takes
  a separate leaf with blank verso,
  \url{http://archive.org/details/practicetypogra11vinngoog} after
  \url{https://en.wikipedia.org/wiki/Half_title}}) has been skipped in
the reprint. The reprint frontmatter has been moved to separate files,
and the bastard title has been scanned from the PIW reprint (ref???).

Moreover the large 6th volume of the dictionary has been split into
two smaller volumes. Again the reprint frontmatter has been moved to
separate files, and the original frontmatter reconstructed.
% Co konkretnie zrobiłem?

The operation actually has been performed later on the DjVu files
after the OCR stage using the DjVu Libre utilities
(\url{http://djvu.sourceforge.net/}), namely djvm.

The results are available at
\url{http://teksty.klf.uw.edu.pl/view/creators/Linde=3ASamuel_Bogumi==0142=3A=3A.html}.

\section{Creating DjVu documents}
\label{sec:creat-djvu-docum}

The DjVu files serve several purposes:

\begin{itemize}
\item They are used for OCR, cf. section \ref{sec:ocr} (page \pageref{sec:ocr}).
\item They are served to the users, cf. ???
\item They are also used in the run time to create the graphic
  snippets, cf. ???. That is the primary reason why the files are
  needed on the Poliqarp server.
\end{itemize}

The DjVu output of didjvu can have bundled or unbundled form. It does
it matter till the stage ???, the bundled form is slightly more
convenient in the early stages.


For this stage we have to prepare two input files:
\begin{itemize}
\item metadata
\item outlines
\end{itemize}
and of course the graphic files with the scans.

\subsection{Converting to DjVu}
\label{sec:converting-djvu}


%  http://korpusy.klf.uw.edu.pl/doc/designing.html#preparing-djvu-sources


didjvu

\url{http://jwilk.net/software/didjvu}


\subsection{OCR}
\label{sec:ocr}

OCR was performed with Tesseract called by \texttt{ocrodjvu} in
several passes with different language parameters. The empty pages
were not subject to OCR.

The results was in the following form:
\begin{itemize}
\item a DjVu document with a hidden text layer
\item hOCR files generated by Tesseract for individual pages
\end{itemize}

The resulting DjVu files are available at \url{http://teksty.klf.uw.edu.pl/20/}.

At this stage it is already possible to create the character histogram
with \url{https://bitbucket.org/jsbien/unihistext/}. The data for the
histogram have to be exported from the resulting DjVu files with
\texttt{djvutxt}.



\subsection{Augmenting the DjVu files}
\label{sec:augm-djvu-files}

There are three ways to augment the DjVu documents:
\begin{itemize}
\item outlines
\item metadata
\item thumbnails
\end{itemize}

The syntax of outlines is described in the man page of djvused.

Outlines may contain both local and external references allowing to
switch easily from volume to volume, therefore the final URLs of the
volume are to be decided at this stage. 

Outlines originally has been prepared with djvusmooth and later
updated by hand. The result was in the form of a djvused script.

djvusmooth can be also used for preparing the
metadata.

Metadata are to be stored in the document, moreover for user
convenience they will be provided also in a different format as corpus
metadata, cf. stage ??? The keywords are not predefined, so the
content can be easily adapted to the needs. It is quite natural to
include the bibliographic description in the metadata.

Thumbnails

\section{Serving the DjVu files}
\label{sec:serving-djvu-files}

The DjVu files should be served unbundled, cf. 

The client references the documents only by their numbers, which are
converted to the appropiate URLs thanks to the
<basename>.djvu.filenames file.
% Szafran - file:// nie działa.

The files can be served before the corpus is created.

\section{Creating poliqarp corpus auxiliary files}
\label{sec:creat-poliq-corp-1}

We need the following input files:

\begin{itemize}
\item tagset definition
\item metadata definition
\item actual metadata
\item structure
\item hOCR:
  \begin{itemize}
  \item generated from the DjVu files, or
  \item generated directly by Tesseract
  \end{itemize}
\item addresses of the DjVu documents 
\end{itemize}

The output consists of augmented hocr files.

The hOCR files are augmented with the structure and in the next stage
only the modified version is used.

\subsection{Preparing tagset for poliqarp}
\label{sec:prep-tags-poliq}

tagset depends on hOCR and XCES files used!


\subsection{Preparing metadata for poliqarp}
\label{sec:preparing-matadata}

The keywords used in the metadata have to be specified in ????.

The actual metadata have the form of a directory tree

????

% http://korpusy.klf.uw.edu.pl/doc/designing.html#preparing-djvu-sources

% Janus s. 47 S/D string date

\textbf{The names of metadata fields are localised later, cf.}

\subsection{Preparing document structure for poliqarp}
\label{sec:prep-docum-struct}

\url{http://korpusy.klf.uw.edu.pl/doc/building.html}


Another configuration file can define the documents structure, which
can be then used in the queries to limit the search. The is the top
level structure specified as page ranges, e.g. frontmatter, preface
etc.  The text units defined this way are called sections.

% jak sekcje są reprezentowane w hOCR? Wilk mi mówił, ale zapomniałem



\section{Creating poliqarp corpus XCES files}
\label{sec:creat-poliq-corp-2}



hOCR files are used indirectly, after converting to XCES.

The ??? is to be used for hOCR generated from the DjVu files and xhocr
tools are to be used for the Tesseract hOCR files. 

The results of converting the Tesseract files to XCES are available at
\url{http://teksty.klf.uw.edu.pl/21/}. The files are in a sense
incomplete, as they don't account for empty pages!

\section{Creating poliqarp corpus}
\label{sec:creat-poliq-corp-3}

We need the following input files created in the previous step:

\begin{itemize}
\item tagset ???
\item metadata
\item actual metadata
\item XCES files
\end{itemize}


veryfing

frequency list:

the script:  marasca-wbl / misc / frequency-list

\section{Creating marasca corpus}
\label{sec:creat-marasca-corp}

We need the following input files created in the previous step:

\begin{itemize}
\item augmented (?) hOCR files
\item addresses of the DjVu documents 
\end{itemize}

augment-djvu-corpus.py newcorpus v01/hocr/v01.hocr v02/hocr/v02.hocr v03/hocr/v03.hocr

Information pages!!!

\section{Running the poliqarp server}
\label{sec:runn-poliq-serv}

Configuration file

\section{Running the marasca server}
\label{sec:runn-marasca-serv}

Configuring www server

\section{Searched texts format}
\label{sec:search-text-format}

For the search engine we need both the page images in the DjVu format
and the OCR results in some form convertible to XCES (XML Corpus
Encoding Standard). As the pages of the documents will be accessed in
random order, we need the so-called unbundled or indirect form of DjVu
documents; this means that every page is stored in a separate file and
can be served independently of others.

\bigskip
Actually two workflows are most practical:
\begin{itemize}
\item Scans and OCR to PDF, PDF to DjVu.  When using an OCR program
  providing output in the PDF format with the text underlying the page
  images, Jakub Wilk's pdf2djvu program will convert them into DjVu
  documents with the hidden text layers. Later appropriate utilities
  will extract the required information.
\item Scans to DjVu, OCR to DjVu and/or hOCR. 
  \begin{itemize}
  \item Scans can be converted to DjVu using the original programs of
    the DjVuLibre bundle (c44 etc.,
    cf. \url{http://djvu.sourceforge.net/}), or the  more recent and more
    powerful didjvu  program by Jakub Wilk, which in particular allows
    to select one of nine additional binarization methods provided by the
    Gamera library
    (cf. e.g. \url{http://gamera.sourceforge.net/doc/html/binarization.html}). At
    \url{http://teksty.klf.uw.edu.pl/6/} you can find for comparison
    the same scans processed with different binarization methods.

    It is worth noting the binary form of the so-called foreground
    (i.e. the proper text without illustration etc.) called stencil is
    inherent in every DjVu document and can be viewed by the user if
    needed.

  \item The DjVu documents obtained from the previous stage may be
    OCRed with Jakub Wilk's ocrodjvu program, which is now
    rather a misnomer: although originally intended to be used with
    the ocropus free OCR program
    (\url{https://github.com/tmbdev/ocropy}), now is primarily used
    with Tesseract (also a free program,
    \url{https://code.google.com/p/tesseract-ocr/}). The output of
    ocrodjvu may be just a DjVu document with the hidden text layer
    added, but it may consist also of the hOCR files produced by the
    OCR engines, which contain additional information not stored in a
    DjVu file; you can find some examples at
    \url{http://teksty.klf.uw.edu.pl/20/}. In consequence, when using
    hOCR we can produce more rich XCES files.

    This is the very workflow we intend to demonstrate at the
    conference, and we will apply it to Linde's dictionary scans
    available at
    \url{http://teksty.klf.uw.edu.pl/view/creators/Linde=3ASamuel_Bogumi==0142=3A=3A.html}.

  \end{itemize}

\end{itemize}

\section{Designing the tagset and describing the document structure}
\label{sec:design-tags-docum}

The OCR results in the hOCR forms can be interpreted as a sequence of
words, i.e. text tokens consisting of character strings, with the
following attributes:
\begin{itemize}
\item Language. For tesseract it is possible to specify several
  languages to be used for recognition. It this worth noting that
  German in contemporary typefaces is considered a different language
  than German printed in Fraktur; although not very logical, it is
  quite convenient in practice.
\item Word recognition confidence, i.e. a numerical value between 0
  and 1.
\item Font properties.
\end{itemize}

To make these properties accessible from a query, it is necessary to
create appropriate XCES tagset. For our demonstration we will used a
tagset which is described by the following Poliqarp configuration
file:

% Daniel Janus

% s. 43-44

% 4.1.1. Część konfiguracyjna
% Plik konfiguracyjny
% Plik konfiguracyjny korpusu ma rozszerzenie .cfg. Można w nim umieszczać komentarze,
% które zaczynają się od znaku # i działają do końca linii. Plik ten jest podzielony na sekcje,
% z których każda zaczyna się od wiersza zawierającego nazwę sekcji ujętą w nawiasy kwadra-
% towe, np. [ATTR], i składa się z szeregu deklaracji postaci nazwa = wartość.
% Istnieje pięć typów sekcji: ATTR, POS, NAMED-ENTITY, ALIASES i STARTUP. Omówimy je po
% kolei.
% Sekcje ATTR i POS definiują tagset używany przez korpus. Pierwsza z nich zawiera listę kate-
% gorii gramatycznych: jako nazwę podaje się nazwę kategorii, a jako wartość — listę możliwych
% wartości tej kategorii oddzielonych spacjami. Na przykład zapis gender = m1 m2 m3 f n de-
% klaruje kategorię rodzaju (ang. gender ) z pięcioma możliwymi wartościami.
% Sekcja POS zawiera listę klas gramatycznych. W tym wypadku nazwa powinna być nazwą
% definiowanej klasy, a jako wartość należy podać listę kategorii, przez które ta klasa się od-
% mienia, oddzielonych spacjami. Lista ta może być w szczególności pusta, co sygnalizuje klasę

% nieodmienną. Ponadto każda kategoria na liście może być ujęta w nawiasy kwadratowe, co
% oznacza, że wartość tej kategorii może, ale nie musi być określona dla segmentu danej klasy
% (np. w Korpusie IPI PAN wartość kategorii akcentowości jest określona tylko dla niektó-
% rych form zaimka). Na przykład zapis praet = number gender aspect [agglutination]
% definiuje klasę pseudoimiesłowów, odmiennych przez liczbę, rodzaj, aspekt i być może aglu-
% tynacyjność.
% Sekcja POS musi wystąpić po sekcji ATTR. Jeśli tej drugiej nie ma, to sekcja POS może
% zawierać wyłącznie definicje nieodmiennych klas gramatycznych. W ten sposób można zdefi-
% niować atomowy tagset, podobny do tych używanych w korpusach języka angielskiego.
% Sekcja NAMED-ENTITY pozwala na określenie alternatywnych nazw (aliasów) dla „nazwa-
% nych bytów” (encji, tzn. kategorii i klas gramatycznych) zdefiniowanych wcześniej. Po lewej
% stronie znaku równości powinna pojawić się nazwa istniejącej encji, a po prawej lista nowych
% nazw oddzielonych spacjami. Oprócz klas i kategorii Poliqarp udostępnia cztery wbudowane
% encje: entity-orth, entity-base, entity-interp i entity-tag, odpowiadające pseudoka-
% tegoriom, o których była mowa w rozdz. 1.2.1 (a zatem te nazwy są pierwotnymi określeniami

% tych pseudokategorii i mogą być używane w zapytaniach zamiast odpowiednio orth, base,
% interp i tag).
% Pewnego wyjaśnienia może wymagać implementacja aliasowania. Poliqarp ma wbudowany
% mechanizm nazywania dowolnych obiektów (konwertowalnych na void *) i wyszukiwania
% ich po nazwie (por. common/entity.h). Oprócz tego z każdym nazwanym obiektem (encją)
% mogą być stowarzyszone pewne metadane — również dowolny wskaźnik — zawierające na
% ogół informację o typie encji. Taka reprezentacja pozwala na łatwe aliasowanie: jeśli wskaźnik
% metadanych wskazuje na samą encję, to reprezentuje ona alias, którego pole danych zawiera
% wskaźnik na aliasowany obiekt.
% I wreszcie sekcja ALIASES umożliwia predefiniowanie dla korpusu innego typu aliasów.
% Alias taki jest parą hnazwa, wartośći, gdzie wartość jest fragmentem zapytania. Jeśli nazwa
% aliasu wystąpi po prawej stronie jakiegoś wyrażenia elementarnego, to w to miejsce zostanie
% podstawiona wartość aliasu. Aliasy takie można tworzyć i usuwać też na bieżąco, podczas
% pracy z korpusem (por. rozdz. 2.2.3 i polecenia protokołu CREATE-ALIAS, DESTROY-ALIAS).
% Dla kompletności opisu należy wspomnieć, że aliasy można też predefiniować w pliku kon-
% figuracyjnym za pomocą alternatywnej składni, która jest wspierana dla zachowania kompa-
% tybilności z wersją wstępną Poliqarpa. Mianowicie sekcja STARTUP może zawierać definicje
% aliasów w postaci1:
% /alias nazwa = wartość1 wartość2 ... wartośćn
% co oznacza alternatywę wartości (a więc, w nowej składni, nazwa = wartość1|...|wartośćn).



\medskip

\noindent
{\it Linde.cfg}
\begin{verbatim}
[attr]
# undefined, Polish, German, Russian:
lang = und pl de ru
# Latin normal, Latin Fraktur, Cyryllic:
script = latn latf cyrl
series = medium bold
shape = upright italic
# word confidence (representation proposed by Jakub Wilk):
wconf = 0 1 2 3 4 5 6 7 8 9

# typically used for the list of part of speech attributes:
[pos]
token = lang script series shape wconf
\end{verbatim}
%
% \noindent
% {\small (The tricky repesentation of confidence has been proposed by Jakub Wilk.)}


\section{Hardware and software requirements}
\label{sec:hardw-softw-requ}

The system has no special hardware requirement, any contemporary
server or even a desktop is sufficient.

The software requires GNU/Linux with a WWW server (tested only with
Apache), the Django framework (https://www.djangoproject.com/) and
Python.

For the demonstration we will use a virtual machine running Debian
GNU/Linux. The machine will be later available for download as an OVA
appliance.


\begin{thebibliography}{4}

\bibitem{JSB2011LNCS} Bień, J.S.: Efficient search in hidden text of
  large DjVu documents. In: Advanced Language Technologies for Digital
  Libraries. Lecture Notes in Computer Science (Theoretical Computer
  Science and General Issues) (6699). Springer, pp. 1-14. ISBN
  978-3-642-23159-9 \url{http://bc.klf.uw.edu.pl/177/}

\bibitem{JSB2014CS} Bień, J, S.: The IMPACT project Polish Ground-Truth texts as
a DjVu corpus. Cognitive Studies | Études Cognitives
(14). pp. 75-84. ISSN 2080-7147.
DOI: http://dx.doi.org/10.11649/cs.2014.008

% \bibitem{JB13dr} Bilińska, J. A.: Analiza i leksykograficzny opis
%   struktury słownika Lindego na potrzeby dygitalizacji. PhD thesis, 2013.
%   \url{http://bc.klf.uw.edu.pl/347/}

\bibitem{etal04} Przepiórkowski, A., Krynicki, Z., Dębowski, Ł.,
  Woliński, M., Janus, D., Bański, P.: A Search Tool for Corpora with
  Positional Tagsets and Ambiguities. In the Proceedings of the Fourth
  International Conference on Language Resources and Evaluation, LREC
  2004,
  pp. 1235--1238. \url{http://nlp.ipipan.waw.pl/~adamp/Papers/2004-lrec/fcqp.pdf}.


\end{thebibliography}

\end{document}

Bień, Janusz S. (2011) Efficient search in hidden text of large DjVu documents. In: Advanced Language Technologies for Digital Libraries. Lecture Notes in Computer Science (Theoretical Computer Science and General Issues) (6699). Springer, pp. 1-14. ISBN 978-3-642-23159-9 
http://bc.klf.uw.edu.pl/177/



Bień, Janusz S. (2014) The IMPACT project Polish Ground-Truth texts as
a DjVu corpus. Cognitive Studies | Études Cognitives
(14). pp. 75-84. ISSN 2080-7147

DOI: http://dx.doi.org/10.11649/cs.2014.008


%%% Local Variables: 
%%% coding: utf-8-unix
%%% mode: latex
%%% TeX-master: t
%%% TeX-PDF-mode: t
%%% End: 
